\documentclass[a4paper,11pt,twoside]{article}
\usepackage{a4}
\usepackage[T1]{fontenc}
\usepackage[latin1]{inputenc}
\usepackage{hyperref}
\usepackage{natbib}
\newcommand{\daisy}{Daisy}
\newcommand{\Daisy}{Daisy}
\newcommand{\cplusplus}%
{{\leavevmode{\rm{\hbox{C\hskip -0.1ex\raise 0.5ex\hbox{\tiny ++}}}}}}
\newcommand{\Cplusplus}{\cplusplus}
\newcommand{\mshe}{Mike/\textsc{she}}
\newcommand{\wintel}{\texttt{win32}}
\newcommand{\dll}{\textsc{dll}}
\newcommand{\Dll}{\textsc{Dll}}
\newcommand{\gui}{\textsc{gui}}
\newcommand{\Gui}{\textsc{Gui}}
\newcommand{\unix}{Unix}
\newcommand{\dhi}{\textsc{dhi}}
\newcommand{\Dhi}{\textsc{Dhi}}
\newcommand{\api}{\textsc{api}}
\newcommand{\Api}{\textsc{Api}}
\newcommand{\lai}{\textsc{lai}}
\newcommand{\Lai}{\textsc{Lai}}
%\newcommand{\url}[1]{\linebreak[4]\texttt{<URL:#1>}}
\renewcommand{\eqref}[1]{Equation~\ref{#1}}

%%% Local Variables: 
%%% mode: latex
%%% TeX-master: t
%%% End: 


\begin{document}

\section*{Estimating the root density distribution from root mass and depth}

In accordance with \cite{gp74}, the root density for a crop can be
described by equation

\begin{equation}
  L_z = L_0\; e^{-a z}
  \label{eq:g+p}
\end{equation}

We here assume that the density is uniformly distributed on the
horizontal plane, an assumption that fails with e.g.\ row crops.

The parameters $a$ and $L_0$ will both vary with time.  For a
production oriented simulation model like
Daisy~\citep{daisy-def,daisy-ems}, it can be more convenient to
specify the density in terms of root dry matter and root depth.  To
achieve this, we assume that the specific root length is a known
constant (rather than varying with depth).  By doing this, we
can find the total root length, given the root dry matter.

\begin{equation}
  l_r = S_r W_r
  \label{eq:root-length}
\end{equation}

\section*{List of symbols}

\begin{tabular}{lll}
Symbol & Unit & Description\\\hline
$a$   & m$^{-1}$ & Root density distribution parameter\\
$l_r$ & m/m$^2$ & Total root length\\
$L_0$ & m/m$^3$ & Root density at soil surface\\
$L_z$ & m/m$^3$ & Root density at soil depth $z$\\
$S_r$ & m/kg & Specific root length\\
$W_r$ & kg/m$^2$ & Total root dry matter\\
$z$ & m & Soil depth. \\
\end{tabular}



\cite{euler83,lambert58}

\cite{lambertwcode}

hroot density component

input 

  Depth, PotRtDpt
  Width, PotRtWdt
  DS
  WRoot

output

  Density

...

Gerwitz and Page

parameters

  SpRtLength
  DensRtTip

Substitute:

     RootLength = WRoot * SpRtLength

Equations:

     L(z) = L0 * exp (-a * z)

     L0 and a are unknown

     integral (L, 0, inf) = RootLength

(1)  L0 * exp (-a * PotRtDpt) = DensRtTip

Solve

    stamfunktionen til L0 * exp (-a * z) er G(z) = -L0 exp (-a z) / a
   
    G(inf) - G(0) =  RootLength

    (-L0 exp (-a inf) / a) - (-L0 exp (-a 0) / a) = RootLength

    L0 / a = RootLength

(2) L0 = RootLength * a

Inds�t (2) i (1)

    RootLength * a * exp (-a * PotRtDpt) = DensRtTip

Substitute x = -a * PotRtDpt => a = -x / PotRtDpt

    - RootLength * x / PotRtDpt * exp (x) = DensRtTip

    x * exp (x) = - DensRtTip * PotRtDpt / RootLength

    x = W (- DensRtTip * PotRtDpt / RootLength)

    a = -W (- DensRtTip * PotRtDpt / RootLength) / PotRtDpt

\addcontentsline{toc}{section}{\numberline{}Bibliography}
\bibliographystyle{elsart-harv}
\bibliography{daisy}

\end{document}
